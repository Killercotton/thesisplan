\chapter{Introduction}
\label{ch:one}

\section{Ideas}
\rro{Ideas for this research project}
\rrotodo{ask RAM members}
\rrotodo{vragen over Linux distributie}

\begin{itemize}
	\item Generic stereo vision module
	\item Solution to a specific problem
	\begin{itemize}
	\item Object recognition
	\item Drone orientation and localisation
	\item Monitoring of a process
	\item Supporting system for sports referee
	\item In combination with SLAM
	\item Control of 3D-printed objects
	\item Stereo vision camera as input for feedback control loop
	\item hand gesture control
	\item facial recognition
	\item 3D reconstruction
	\item generate 3D model for printer
	\item research on how to use stereo vision as input for control loop like SHERPA arm or drone(position of camera, algorithms, fps)
	\end{itemize}
\end{itemize}

\section{Questions}
\begin{itemize}
	\item How relevant is stereo vision in a research context?
	\item How relevant is stereo vision for RaM?
	\item What is the topic of ongoing research in stereo vision?
	\item What kind of problems are solvable with stereo vision?
	\item How much literature do I have to read to know what the current state of research is?
	\item Would it be interesting to publish an article presenting the results?
	\end{itemize}

\section{People to talk to}
\begin{itemize}	
	\item Ferdi van der Hejiden, RaM, image processing
	\item Han Wopereis, RaM, Aeroworks
	\item Klaas Jan Russcher, RaM, Rove
	\item Eamon Barrett, RaM, Sherpa (maybe not necessary, already talked with other sherpa member)
	\item Bert Molenkamp, CAES
	\rrodone{afspraak met Bert}
	\item Johan Engelen, RaM, Inspectie robots
	\rrodone{afspraak met Johan}
	\item Stefano Stramigioli, RaM, all projects
	\item Douwe Dresscher, RaM, ROSE
\end{itemize}

\section{Context}
\rrotodo{Read more literature}
\rrotodo{Research capabilities of RaMStix}
\rro{context for stereo vision and posible application}
A stereo vision module could be interesting for the SHERPA project. The rover has an arm that needs to pick up a small quadcopter
to change its battery. Stereo vision could be used to detect the drone without the necessity for markers on the drone. Additionally, 
it would be possible to track the distance between arm and drone. A first estimation of requirements would be to have a framerate 
between 10 and 30 to detect a drone in (slow) motion. The used cameras should be fitted with a global shutter to get pictures with sufficient quality. A problem could be missing computation power of the RaMStix when more sophisticated image processing is necessary. For a stereo vision implementation in general, the existing bus could be a bottle neck. 

\section{Research question}
\rrotodo{decide on what to do}
\rro{specific research or problem I want to solve}
\rro{how efficient/effective is my solution?}
\rro{how appropriate is my solution?}
\section{Objective}
\rro{desired outcome}

\section{Approach}
\rro{how to solve problem}

\section{Report outline}
\rro{outline of this document}

